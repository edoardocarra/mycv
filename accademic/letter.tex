\documentclass[acmlarge]{acmart}

% Use the "authoryear" citation style, and make sure citations are in [square brackets].
\citestyle{acmauthoryear}

\usepackage[ruled]{algorithm2e} % For algorithms
\renewcommand{\algorithmcfname}{ALGORITHM}
\SetAlFnt{\small}
\SetAlCapFnt{\small}
\SetAlCapNameFnt{\small}
\SetAlCapHSkip{0pt}

\settopmatter{printacmref=false} % Removes citation information below abstract
\renewcommand\footnotetextcopyrightpermission[1]{} % removes footnote with conference information in first column

\hypersetup{draft}
\begin{document}

% Title. 
% If your title is long, consider \title[short title]{full title} - "short title" will be used for running heads.
\title{Of all to the people who are gonna tell you "take meee, i'm skilled, i write sickeningly performant code with both the hands tied and the eye closed", why you should choose me?}
\maketitle

First of all thank you for reading this letter. 
My points will be not very strong, since my background is only accademic until now. I'm strongly motivated to learn the industry, and make myself fit at best inside it. I want true artist use our tools,  I can give to you discipline, and respect for achieving goals in time and in the exact way they were supposed to be achieved. 


Dear reviewers,
but all i did
we write this letter to highlight the differences in this resubmission of our work.
The main improvements in this submission are in response to the requests made 
in the summary review, namely to perform direct comparisons withe prior work,
validate our results with a user study, provide more details on the implementation and
clarify our technical contributions. 
We believe we have address all these issues as summarized below.

\begin{enumerate}

\item \textbf{Comparisons (Sect. 7)} 
For the comparison with prior work, we stress that to the best of knowledge 
there is no other system that can version whole 3D environments. Since we were 
not asked for a specific comparison, we chose the prior systems that we felt
are more related to our work. We compare to MeshGit [Denning and Pellacini, 2013]
for polygonal meshes, JsonDiffPatch for scene nodes, Git for generic version 
control, and Plastic SCM since it has been adopted in industry for versioning 
game scenes and asset. Compared to these systems, we are significantly more 
efficient, scale better, and, more importantly, our system was the only one that 
could reliably merged all edits without incurring in mis-detected conflicts.
Said another way, we show that our system is the only one that can version 3D 
scenes reliably.

\item \textbf{User Study (Sect. 7)}
We also validated our system with a user study where we simulate a collaborative 
workflow. Users performed free form edits to a few scenes, and used
our system both to view changes using highlights and to merge. In both cases,
our system was rated as very high, showing that our work will likely have 
uses in collaborative workflows and foster future work in this area.

\item \textbf{Technical Contributions (Sect. 1)}
We stress here that this work is a system paper. In our opinion, the hallmark 
of a good system paper is to define a practical problem in our field (Sect. 1), 
present and motivate a system design that solves that problem (Sect. 3, 4 and 5),
and analyze the system performance in non-trivial cases (Sect. 6, 7). When pressed 
to list specific points of technical contribution, we believe we have 
introduced the first data structure for scene versioning and reliable algorithms 
for diffing and merging algorithms, as listed in Sect 1. 
We want to stress though that we believe that our main contribution is in 
characterizing the version control problem for 3D scenes, presenting an overall 
system design to address it, and showing a solution that works reliably 
on different scenes and edits.
Again, to the best of our knowledge, there is no other system, in the literature
or commercial, that can handle the edits in this paper.

\item \textbf{Paper Presentation (Sect. 5, Supplemental)}
We include several small changes suggested by reviewers together with 
pseudo-code for the underlying algorithms to give more details on the actual 
implementation. The pseudo-code is included in supplemental to save space. 
If the paper were to be accepted, we will release a open source
implementation too, that is not included here to maintain anonymity.

\item \textbf{System Implementation (Sect. 6)}
We further improved our implementation by using better compression methods
for the final repository and further tested the system on a larger forest
environment (Fig. 8).

\end{enumerate}

\end{document}
