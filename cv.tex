%%%%%%%%%%%%%%%%%%%%%%%%%%%%%%%%%%%%%%%%%
% Compact Academic CV
% LaTeX Template
% Version 1.0 (10/6/2012)
%
% This template has been downloaded from:
% http://www.LaTeXTemplates.com
%
% Original author:
% Dario Taraborelli (http://nitens.org/taraborelli/home)
%
% License:
% CC BY-NC-SA 3.0 (http://creativecommons.org/licenses/by-nc-sa/3.0/)
%
% Important:
% This template needs to be compiled using XeLaTeX
%
% Note: this template has the option to use the Hoefler Text font, see the
% font configurations section below for instructions on using this font
%
%%%%%%%%%%%%%%%%%%%%%%%%%%%%%%%%%%%%%%%%%

%----------------------------------------------------------------------------------------
%	PACKAGES AND OTHER DOCUMENT CONFIGURATIONS
%----------------------------------------------------------------------------------------

\documentclass[10pt, a4paper]{article} % Document font size and paper size

\usepackage{fontspec} % Allows the use of OpenType fonts

\usepackage{geometry} % Allows the configuration of document margins
\geometry{a4paper, textwidth=6in, textheight=10in, marginparsep=7pt, marginparwidth=.6in} % Document margin settings
\setlength\parindent{0in} % Remove paragraph indentation

\pagenumbering{gobble}

\usepackage[usenames,dvipsnames]{xcolor} % Custom colors

\usepackage{sectsty} % Allows changing the font options for sections in a document
\usepackage[normalem]{ulem} % Custom underlining
\usepackage{xunicode} % Allows generation of unicode characters from accented glyphs
\usepackage{tikz}
\defaultfontfeatures{Mapping=tex-text} % Converts LaTeX specials (``quotes'' --- dashes etc.) to unicode

\usepackage{marginnote} % For margin years
\newcommand{\years}[1]{\marginnote{\scriptsize #1}} % New command for including margin years
\renewcommand*{\raggedleftmarginnote}{}
\setlength{\marginparsep}{7pt} % Slightly increase the distance of the margin years from the contant
\reversemarginpar

\usepackage[xetex, bookmarks, colorlinks, breaklinks, pdftitle={Edoardo Carra - vita},pdfauthor={Edoardo Carra}]{hyperref} % PDF setup - set your name and the title of the document to be incorporated into the final PDF file meta-information
\hypersetup{linkcolor=blue,citecolor=blue,filecolor=black,urlcolor=MidnightBlue} % Link colors

%----------------------------------------------------------------------------------------
%	FONT CONFIGURATIONS
%----------------------------------------------------------------------------------------

% Two font choices are available in this template, the default is Linux Libertine, available for free at: http://www.linuxlibertine.org while the secondary choice is Hoefler Text which comes bundled with Mac OSX.
% To use Hoefler Text, comment out the Linux Libertine block below and uncomment the Hoefler Text block. You will also need to replace the "\&" characters with "\amper{}" in section titles.

% Linux Libertine Font (default)
\setromanfont [Ligatures={Common}, Numbers={OldStyle}, Variant=01]{Linux Libertine} % Main text font
%\setmonofont[Scale=0.8]{Monaco} % Set mono font (e.g. phone numbers)
\sectionfont{\mdseries\upshape\Large} % Set font options for sections
\subsectionfont{\mdseries\scshape\normalsize} % Set font options for subsections
\subsubsectionfont{\mdseries\upshape\large} % Set font options for subsubsections
\chardef\&="E050 % Custom ampersand character

% Hoefler Text Font (bundled with Mac OSX)
%\setromanfont [Ligatures={Common}, Numbers={OldStyle}]{Hoefler Text} % Main text font
%\setmonofont[Scale=0.8]{Monaco} % Set mono font (e.g. phone numbers)
%\setsansfont[Scale=0.9]{Optima Regular} % Set sans font, used in the main name and titles in the document
%\newcommand{\amper}{{\fontspec[Scale=.95]{Hoefler Text}\selectfont\itshape\&}} % Custom ampersand character
%\sectionfont{\sffamily\mdseries\large\underline} % Set font options for sections
%\subsectionfont{\rmfamily\mdseries\scshape\normalsize} % Set font options for subsections
%\subsubsectionfont{\rmfamily\bfseries\upshape\normalsize} % Set font options for subsubsections

%----------------------------------------------------------------------------------------

\begin{document}

%\begin{tikzpicture}[remember picture, overlay]
%  \node [anchor=north east, inner sep=100pt]  at (current page.north east)
%     {\includegraphics[height=2.5cm]{cv.jpg}};
%\end{tikzpicture}

%----------------------------------------------------------------------------------------
%	CONTACT AND GENERAL INFORMATION SECTION
%----------------------------------------------------------------------------------------

{\LARGE Edoardo Carra}\\ % Your name

Phone: \texttt{+39 3319733470}\\ % Your phone number
%Fax: \texttt{609-924-8399}\\[.2cm] % Your fax number
Email: \href{mailto:edo.carra@gmail.com}{edo.carra@gmail.com}\\ % Your email address
Skype: \href{https://join.skype.com/invite/gQN9qtZGLZ7r}{carra\_127}\\
LinkedIn: \href{https://www.linkedin.com/in/edoardocarra/}{https://bit.ly/2KwFBBK}\\ % Your academic/personal website

%------------------------------------------------

Born: July 16, 1991---Rome, Italy\\ % Your date of birth
Nationality: Italian\\ % Your nationality
Languages: Italian mother tongue, english fluency

%------------------------------------------------
\section*{About me}

I love creating tools for digital artists. During the PhD i learned to prototype and implement systems always considering out-of-core 3D environments as test case. In the past years i gained some experience in web-oriented graphics, designing and implementing methods for delivering computationally demanding 3D environments on a browser window. I mostly code in C++ and Python, but i can easily switch on web-oriented technologies. I like long walks, big beers, and talking (a lot) about food.

%----------------------------------------------------------------------------------------
%	EDUCATION SECTION
%----------------------------------------------------------------------------------------

\section*{Education}

\years{2016 - now} \textsc{PhD} in Computer Graphics, Sapienza University of Rome, [supervisor: Fabio Pellacini] \emph{"Multi-user collaboration in digital content creation"}\\ % Your current or previous employment position
\years{2016}\textsc{MSc} in Engineering in Computer Science, Roma Tre University, \emph{"Production techniques and tools for creating photorealistic environments over web"}

%----------------------------------------------------------------------------------------
%	SKILLS SECTION
%----------------------------------------------------------------------------------------

\section*{Technical skills}

 \textbf{programming language} C++, Python $\bullet$ \textbf{web} JavaScript, TypeScript, NodeJS, HTML5, CSS3 $\bullet$ \textbf{source control} Git $\bullet$ \textbf{computer graphics api} Vulkan, OpenGL $\bullet$ \textbf{3d format} glTF, FBX, OBJ $\bullet$ \textbf{database} PostgreSQL, MongoDB $\bullet$ \textbf{operating system} Windows, macOS and Linux (Bash scripting) $\bullet$ \textbf{tools} CMake, Vim, FFmpeg, ImageMagick $\bullet$ \textbf{software} Unity, Blender, Affinity Designer

%----------------------------------------------------------------------------------------
%	WORK EXPERIENCE SECTION
%----------------------------------------------------------------------------------------

\section*{Appointments held}

\years{2019} Tutor, Physics Programming, Masters degree course in Computer Game Development, University of Verona\\
\years{2016} Assistant Tutor, Computer Graphics course, \textsc{MSc} in Engineering in Computer Science, Roma Tre University\\
\years{2012} Developer, Sira srl, Rome

%----------------------------------------------------------------------------------------
%	PUBLICATIONS AND TALKS SECTION
%----------------------------------------------------------------------------------------

\section*{Publications}

\years{2019} Edoardo Carra, Christian Santoni, Fabio Pellacini, \emph{Grammar-based procedural animations for motion graphics}, Computers \& Graphics, Volume 78, February 2019, Pages 97-107\\
\years{2018} Edoardo Carra, Christian Santoni, Fabio Pellacini, \emph{gMotion: A Spatio-Temporal Grammar for the Procedural Generation of Motion Graphics}, Proceedings of Graphics Interface 2018: Toronto, Ontario, May 8 - 11 2018\\
\years{2016} Federico Spini, Enrico Marino, Michele D'Antimi, Edoardo Carra, Alberto Paoluzzi, \emph{Web 3D indoor authoring and VR exploration via texture baking service}, Proceedings of the 21st International Conference on Web3D Technology: Anaheim, California, July 22 - 24 2016

%----------------------------------------------------------------------------------------
%	GRANTS, HONORS AND AWARDS SECTION
%----------------------------------------------------------------------------------------

\section*{Fundings}

\years{2018} Funding for scientific research, Sapienza Research Calls\\
\years{2016} Funding from Sogei S.p.A, Italian Ministry of Economy and Finance\\


\vfill{} % Whitespace before final footer

%----------------------------------------------------------------------------------------
%	FINAL FOOTER
%----------------------------------------------------------------------------------------

\begin{center}
{\scriptsize I hereby authorize the use of my personal data in accordance to the GDPR 679/16.} % Any final footer text such as a URL to the latest version of your CV, last updated date, compiled in XeTeX, etc
\end{center}

%----------------------------------------------------------------------------------------

\end{document}